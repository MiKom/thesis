\chapter{Introduction}

%This chapter describes motivations behind the development of the thesis. It
%first briefly describes the challenges that arise with ever-increasing capacity
%of computer hardware, that enables nearly photorealistic quality in real-time graphics, but
%at the same time requires enormous amounts of artistic content like models and textures. Procedural
%generation helps to overcome this problem by the usage of parametrized
%algorithms that take limited user input and generate content usually by means
%of randomization or simulation of physical phenomena.
%
%One of such types of content, that may be desirable to generate is a karst cave.
%This geological structure may be an interesting setting for a movie scene or a
%computer game level. At the same time, it is a very complex structure
%that may be time-consuming to model by hand in a way that is both aesthetically
%pleasing and, at least partially, physically correct.

%\section{Advancement of graphics-generation hardware}

%Capabilities of modern computer graphics hardware enables it to


Karst formations are ubiquitous in every continent of Earth. It's estimated that
about 25\% of Earth population depends on drinking water obtained from karst
aquifers \parencite{ford2007karst}. With such profound infulence on human race,
it is essential to know how these geological structures evolve and how they may
react to human activity.

Various simulation models were developed that try to predict how karst aquifers
evolve in time, and how they react to changes in environment. These models are
implemented in computer software and represent simulated karst structure as
net of fractures.

These tools, being aimed at speleogenesis experts, present results of
calculations with simple plots. Programming project of this thesis called \emph{karstgen} provides
solution for richer presentation of geometric structure of modelled karst
formation. It can take input data in format that is similar to formats of files
produced by simulation software and generate triangle mesh in two file formats,
one of which is simple and popular textual file format supported by most
three--dimensional modelling software. Presentation in such program may be
beneficial for better understanding of data or for later usage in e.g. video
games.

Since karst evolution models usually simulate large datasets, karstgen uses
GPU acceleration to speed--up mesh generation process.

\pagebreak
\section{Structure of the thesis}

Below is overview of each chapter along with description on how it contributes to
the thesis.
\begin{description}
  \item[\Cref{chap:karstification} -- Karst and karstification process] \hfill \\
    This chapter introduces basic definitions related to karst landscape forms
    used later in the work. Karstification process is presented with basic
    overview of chemical reactions that drive it. Information contained in this
    chapter provides rationale for the shape of karst evolution simulation
    models presented in subsequent chapters.
  \item[\Cref{chap:related} -- Related work] \hfill \\
    References to various works relevant to the subject are presented here.
    Brief overview of karst evolution models is presented what explains some
    of the design decisions taken in the programming project.

    Several works related to rendering, rather than simulating, caved terrains
    are also described.
  \item[\Cref{chap:cl} -- OpenCL heterogeneous programming platform] \hfill \\
    Here, OpenCL is described. It's a programming library, maintained by Khronos
    Group Inc., that lets programmers leverage diverse computational resources
    offered by modern computers in standardized manner. OpenCL is extensively
    used by programming project of this thesis so this introduction gives
    reader a background for understanding implementation details.
  \item[\Cref{chap:marchingcubes} -- Isosurface extraction with Marching Cubes] \hfill \\
    In this chapter Marching Cubes algorithm that is used in programming project
    is described. This algorithm extracts polygonal meshes of isosurfaces from
    three--dimensional scalar functions. GPU--accelerated variant used in
    project is also described.
  \item[\Cref{chap:project} -- Programming project description] \hfill \\
    This chapter describes features and architecture of programming project, as
    well as some more in--depth technical details of the implementation.
  \item[\Cref{chap:furtherwork} -- Conclusions and further work] \hfill \\
    Final chapter of the thesis summarizes results of the programming project.
    Possible use cases are presented. Potential new areas of improvement and
    further development are discussed as well.
\end{description}

