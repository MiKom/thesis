\chapter*{Introduction}

%This chapter describes motivations behind the development of the thesis. It
%first briefly describes the challenges that arise with ever-increasing capacity
%of computer hardware, that enables nearly photorealistic quality in real-time graphics, but
%at the same time requires enormous amounts of artistic content like models and textures. Procedural
%generation helps to overcome this problem by the usage of parametrized
%algorithms that take limited user input and generate content usually by means
%of randomization or simulation of physical phenomena.
%
%One of such types of content, that may be desirable to generate is a karst cave.
%This geological structure may be an interesting setting for a movie scene or a
%computer game level. At the same time, it is a very complex structure
%that may be time-consuming to model by hand in a way that is both aesthetically
%pleasing and, at least partially, physically correct.

%\section{Advancement of graphics-generation hardware}

%Capabilities of modern computer graphics hardware enables it to
