\chapter{Programming project description}
\label{chap:project}
\section{Introduction}

Programming project of this thesis is a set of command line programs, collectively
called \emph{Karstgen}, that take
the description of karst cave fracture net and generate polygon mesh in
simple and popular Wavefront OBJ textual file format\footnote{\url{http://www.martinreddy.net/gfx/3d/OBJ.spec}}.

Models created this way may be opened in 3D editing program for further editing
and examination.

Karstgen can also create models for \emph{Vorticity} game engine that was created
by the author together with mr Michał Siejak for graphics related courses\footnote{Computer
  Graphics and Visualiation, summer semester 2009/2009 and Group Project, summer
semester 2009/2010} during licenciate studies at Adam Mickiewicz University \todo{check form}of
Poznań.

\section{Architecture}

Karstgen was created with Unix Philosophy in mind \parencite{raymond2003art}.
It is made of two programs named \emph{blobber} and \emph{mcblob} that have
clearly defined reposinsibilities and communicate through simple textual data
format.  Both programs may take input either from files or from standard input
so they can be piped together with shell pipes. Data flow of karstgen is
presented in \autoref{fig:karstgenflow}.
\begin{figure}[ht]
  \begin{center}
    \includegraphics[width=\textwidth]{chapters/project/karstgenflow.jpg}
  \end{center}
  \caption{Data flow of karstgen program. Converted from some other possible
  format is not part of the project.}
  \label{fig:karstgenflow}
\end{figure}
\todo{replace \autoref{fig:karstgenflow} with TikZ graphics}

\subsection{Blobber}
Blobber takes description of a fracture net in a simple textual format and
generates list of metaballs (see \autoref{sub:metaballs}). It can
optionally tilt positions and sizes of metaballs in random but adjustable manner
for more natural--looking results. Blobber also controls quality of the final
geometry. For information about runtime parameters invoke
\begin{verbatim}
./blobber --help
\end{verbatim}

\subsection{Mcblob}
Output generated by blobber is consumed by program named \emph{mcblob}\footnote{Marching
Cubes from blobs} that is in fact a general purpose tool that may be used to
generate geometry from a list of metaballs in 3D space.
\section{Implementation}
\subsection{Metaballs}
Implementation heavily relies on rendering with metaballs. Metaball is a scalar
function in the form:
\begin{equation}
  f(x,y,z)=\frac{d}{(x-x_0)^2+(y-y_0)^2+(z-z_0)^2}
  \label{eq:metaball}
\end{equation}
where $(x_0,y_0,z_0)$ is the center of the metaball and $d$ is a multiplier that
controls the size of the metaball.

If more than one metaball is present in the scene, density function (see Definition
\autoref{def:density function})
is in the form:
\begin{equation}
  d(x,y,z) = \sum_{i=0}^{n} f_i(x,y,z)
  \label{eq:metaballdensity}
\end{equation}
where $n$ is the total number of metaballs in the scene and $f_i$ is function~\ref{eq:metaball}
of the $i$-th metaball.

Metaballs were discovered by Jim Blinn when he was working on visualisation of
molecular structures \parencite{Blinn:1982:GAS:357306.357310}.
\todo{add input file to thesis}
\label{sub:metaballs}
\begin{figure}[htb]
  \begin{center}
    \includegraphics[width=\textwidth]{chapters/project/metaballs.png}
  \end{center}
  \caption{Two metaballs at various distances showing how they are ,,melitng''
    together when getting closer to each other. Geometry was generated with
    \emph{mcblob} program and final image was rendered with Blender 2.68 with
    Cycles renderer. Input file for mcblob that generates this is included with the thesis
      in \texttt{mcblob.in} file.
  }
  \label{fig:metaballs}
\end{figure}

They have the interesting property of ,,melting'' when getting close together
forming structure with somewhat ,,organic'' look and feel (see \autoref{fig:metaballs}).

\subsection{Overview}
Both blobber and mcblob are implemented in C++ language with latest C++11
version of the standard. Build system used to compile the code is \emph{CMake}\footnote{\url{http://cmake.org}}
-- cross platform make that can generate native projects for various IDEs.
Executables use Boost Program Options\footnote{\url{http://www.boost.org/doc/libs/1\_54\_0/doc/html/program\_options.html}}
library for parsing command line arguments and providing help.

Karstgen uses unit testing framework \emph{Google~Test}\footnote{\url{http://code.google.com/p/googletest/}}.
Documentation is automatically generated from sources with Doxygen\footnote{\url{http://doxygen.org}}
tool.

\subsection{Blobber}


\begin{lstlisting}
struct DataPoint
{
	int x, y, z;
	float midDiam;
	std::vector<float> xData;
	std::vector<float> yData;
	std::vector<float> zData;
};
\end{lstlisting}
\subsection{Mcblob}
