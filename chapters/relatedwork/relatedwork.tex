\chapter{Related work}
\label{chap:related}

This chapter presents basics of modelling of processes governing evolution of
karst aquifers. This description will provide basis for the data formats
used in programming project. Some visualisation techniques used previously for
caved terrains will also be presented.

\section{Modelling karst aquifers}

As karst aquifers contain network of fractures (see \autoref{chap:karstification})
a simulation of flow and chemical reactions in single fracture is the basic
building block of presented models.

These fractures are connected to form larger, two or three--dimensional networks,
that represent whole conduits.

First attempts to describe processes taking place during evolution of karst
aquifers with numerical models took place in early 1980's \parencite[p. 3]{hiller2013}.
\Cite{Buhmann1985189} developed numerical model for ternary chemical system
(\cee{CaCO3}--\cee{CO2}--\cee{H2O}) in open systems (where \cee{CO2} is exchanged
with atmosphere) and for closed ones \parencite{Buhmann1985109}.

\subsection{Single fracture simulation}

With these dissolution models in place several models of single conduit were
presented \parencite[pp. 4]{hiller2013}.

Single fracture is modelled as a circular conduit in the intersection
of fissure and bedding plane \parencite{Kaufmann200962} or as a space between
two parallel walls of rock separated by aperture width \parencite{dreybrodt2002}.

\subsection{Two--dimensional simulations}

Single conduit one--dimensional models were later expanded into second dimension
by combining set of conduits into a connected network \parencite[pp. 4--5]{hiller2013}.
In such networks, fractures are organized into uniform, regular structure.

\subsection{Three--dimensional simulations}

With more powerful computational resources researchers started research on
three--dimensional models that could finally provide insight into evolution
of real--life karst aquifers. Such models were proposed by
\cites{annable2003}{WRCR:WRCR9525}{Kaufmann2010241}.

Work by \cite{hiller2013} summarizes current state of three--dimensional models.
His thesis contains overview of modelling techniques and approaches. Simulation
of real--live karst aquifer near dam--site is presented that matches
observations. This shows validity of proposed models.

\section{Visualisation techniques}

Rendering techniques that touched the issue of cave rendering never tried to
provide both physical accuracy and visually appealing graphics.

In \cite{gpugems3ch01} a method for rendering procedural terrains is described
that in some circumstances can generate caves. Similar to this work, presented
method uses Marching Cubes algorithm (see \autoref{chap:marchingcubes})
implemented on GPU (albeit in shaders, and not with any GPGPU solution) and
carefully crafted density functions based on noise sampling.

\Cite{forstmann2005} proposed method of on--the--fly rendering of procedural
terrains that may also contain caves. It uses hierarchy of nested clip--boxes
for LOD\footnote{Level of detail} calculation to reduce workload of the GPU.
