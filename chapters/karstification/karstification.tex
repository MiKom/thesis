\chapter{Karst and karstification process}
\section{Introduction}
This chapter will briefly describe karst and processes that govern the
development of karts caves.
\todo{Expand when chapter finished}
\section{Basics}

\subsection{Definitions}
\emph{Karstification} is not a strictly defined term. Depending on context it may
mean all forms of corrosion of soluble rocks or it may encompass whole range of
processes that lead to devolopment of karst formations.

Usually karstification means a landscape forming process that consists of dissolution
of various kinds of bedrock. The most common kinds of solutes are limestone,
dolomite, and gypsum \parencite{karstglossary}. However, given right conditions
even some weathering-resistant rocks like quartzite may be subject to 
karstification \parencite{migon2010}\todo{Check reference}.

Although chemical dissolution is the main driving force behind karstification,
mechanical forces may also play a role in the final looks of karst landscape.
That's why sometimes, all these forces together are put under the umbrella term
of karstification.

\emph{Karst} is a terrain formation developed throught means of
\emph{karstification}. The origin of the term is a German form of Slavic word
kras or krš meaning bleak, waterless place.

\subsection{Elements of karst landscape}

Karstification process may produce very interesting and varied landscape. Most
common elements of karst landscape will be shown and described.

\begin{figure}
  \centerline{\includegraphics[width=480px]{chapters/karstification/karst_landscape.jpg}}
  \caption{Karst landscape showing various features of karst aquifers.
    Figure from \cite{marshak2006}}
\end{figure}

\todo{Write about karst formations}

\section{Overview of karstification process}

Most important factor in karstification process is flow of solvent throught an
underground aquifer. Since bedrock is subject to geological processes, a
net of fractures of varying diameter and shape is present in it. Water that flows
through this kind of net reacts with rock in ways described later.

The inflow of water to the system may come from precipation, rivers or lakes.

There are also cases of ground formation resulting from human activity.
Although not truly a karst process, a sinkhole that opened in 2010 in city of
Guatemala was a result of combination of loose ground made of volcanic ash and
inadequate draining system, that couldn't dissipate large amounts of water
brought by tropical storm Agatha  \parencite{times2010}.



\section{Limestone dissolution}

\section{Formation of speleothems}
